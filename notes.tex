\documentclass{article}
\usepackage{fullpage}
\usepackage{hyperref}

\title{\vspace{-3ex}notes}
\date{2013/06/17---}

\begin{document}
\maketitle
\url{http://www.github.com/eqy/reu}

Insights and observations found in papers as well as useful definitions for
jargon and a cheat-sheet for acronyms. The annotations in the bibliography could
be an adaptation of these.

\section{Jargon Cheat-Sheet}
\begin{itemize}
\item \textbf{Ablation}: Like resection, but at the surface
\item \textbf{Gliosis}: Glial cell hypertrophy
\item \textbf{Hypovolemia}: Decreased blood volume
\item \textbf{Ictal}: During a seizure/attack
\item \textbf{Interictal}: Between seizures/attacks
\item \textbf{Multilayer Perceptron}: Type of Neural Network used in supervised
\item \textbf{Refractory}: Resisting control, not responding [to medication] 
\item \textbf{Resection}: Removal of a part of an organ e.g. Left temporal
resection as treatment for epilepsy
ML\footnote{Wikipedia article on this has a ``factual dispute''... sticking with
rudimentary description for now}
\item \textbf{West Syndrom}: Infant eplipsy, often leading to other development
impairments
\end{itemize}

\section{Acronym Cheat-Sheet}
\begin{itemize}
\item \textbf{AED}: Antiepileptic drugs
\item \textbf{CADT}: Computer aided diagnostic tools
\item \textbf{CL1OCV}: Cyclical leave-one-out cross validation
\item \textbf{EEG}: Electroencephalography
\item \textbf{FDG-PET}: 18F-Fludeoxyglucose Positron Emission Tomography
\item \textbf{ML}: Machine Learning
\item \textbf{NNP}: Neurologicaly normal patients
\item \textbf{PWE}: Epileptic seizure patients
\item \textbf{PWN}: Non-epileptic seizure patients
\item \textbf{RFTC}: Random field teory correction
\item \textbf{ROI}: Region of interest
\item \textbf{R/LTLE}: Right/Left Temporal Lobe Epilepsy
\item \textbf{TBI}: Traumatic Brain Injury

\end{itemize}

\section{Epilepsy: Common}
\subsection{Henry--Presurgical epilepsy localization with interictal cerebral
dysfunction}
This paper documents a variety of scenarios with respect to ictal, interictal,
and peri-ictal hypometabolism in TLE patients. 

\textbf{Background}: ``Neuropsychological evaluation is critical to evaluating
thepotential benefit of epilepsy surgery''

\begin{itemize}
\item ``Bilateral but assymetric temporal lobe hypometabolism can be
detected readily with statistical parametric imaging techniques, but usually
appears only as unilateral temporal hypometabolism on visual interpretation.''
\item AEDs may uniformly reduce metabolism. 
\item Higher resolution better detect hypometabolism.
\item Hypometabolism is diffiuse
\item Temporal hypometabolism is usually more severe than extratemporal
hypometabolism
\item Hypometabolism changes with time following seizures
\item ``Symmetric, severe, bilateral temporal hypometabolism also is associated
with a higher incidence of postoperative seizures''
\item Infants suffering from West's syndrome that underwent resection ``often
had cessation of seizures, and normal or near-normal cognitive development,'' as
opposed to developing autism

\end{itemize}

\subsection{Kerr--Balancing Clinical and Pathologic Relevence in the Machine Learning
Diagnosis of Epilepsy}
\textbf{Distinction}: Patients with non-epileptic seizures as control versus
normal patients as control.

``Non-epileptic seizures are primarily psychiatric events.'' ``...based on
ruling out all organic causes for the attacks.''


Common features of Epilepsy: ``Hypersynchronized activity of neurons, inhibited
activity in surrounding tissue, focal cortical atrophy, MRI signal
hyperintensivity, increased cell death of inhibitory cells, ensuing gliosis''

\textbf{Insight}: Patients with seizures may recieve antiepileptic drugs that manipulate
neural networks and therefore EEGs that differences between NNPs and PWEs are
enhanced.

\textbf{Observation}: PWNs and PWEs may share many common underlying factors: ``PWNs
model their seizures after those they have seen or heard about...'' and ``both
PWN and PWE are associated with traumatic brain injury''

\textbf{Insight}: Comparison between PWN and PWE models the comparison physicians
perform--that is, if a patient suffers from seizures, is this seizure epileptic
or non-epileptic? This contrasts with trying to determine whether a patient is
epileptic regardless of whether the patient suffers from seizures.

\textbf{Methodology}: Multilayer Perceptron used to ``discriminate between either PWN or
NNP versuses RTLE or LTLE''. \textbf{minimum redundancy-maximum relvancy mRMR of
MATLAB}.

\textbf{Observation}: Each binary comparison e.g. LTLE vs. PWN, LTLE vs. NNP, RTLE vs.
NNP, etc. often corresponds to differing ROIs. Even LTLE vs. PWN and LTLE vs.
NNP have differences--PWN and NNP differ from LTLE differently!

\textbf{Observation}: Despite the differences in ROIs for different control
groups, the ability to diagnose LTLE/RTLE remained similar. Therefore,
comparing PWEs to NNPs does \textbf{not} artificially increase discriminative
ability!

\textbf{Take Home}: Let the clinical question determine the control group, and
the question may be a ``pathologic description of disorders'' or the
``development of clinically applicable tools.''  

\end{document}
