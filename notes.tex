\documentclass{article}
\usepackage{fullpage}
\usepackage{hyperref}

\title{\vspace{-3ex}notes}
\date{2013/06/17---}

\begin{document}
    \maketitle
    \url{http://www.github.com/eqy/reu}

    Insights and observations found in papers as well as useful definitions for
    jargon and a cheat-sheet for acronyms. The annotations in the bibliography 
could be an adaptation of these.

    \section{Jargon Cheat-Sheet}
        \begin{itemize}
            \item \textbf{Ablation}: Like resection, but at the surface
            \item \textbf{Cortical Dysplasia}: \item \textbf{Gliosis}: Glial 
cell hypertrophy
            \item \textbf{Hypovolemia}: Decreased blood volume
            \item \textbf{Ictal}: During a seizure/attack
            \item \textbf{Idiopathic}: Of indeterminate cause
            \item \textbf{Interictal}: Between seizures/attacks
            \item \textbf{Multilayer Perceptron}: Type of Neural Network used in 
supervised
            \item \textbf{Prior}: Prior probability distribution
            \item \textbf{Refractory}: Resisting control, not responding [to 
medication] \item \textbf{Resection}: Removal of a part of an organ e.g. Left 
temporal
            resection as treatment for epilepsy
            ML\footnote{Wikipedia article on this has a ``factual dispute''...  
sticking with
            rudimentary description for now}
            \item \textbf{West Syndrom}: Infant epilepsy, often leading to other 
development
            impairments
        \end{itemize}

    \section{Acronym Cheat-Sheet}
        \begin{itemize}
            \item \textbf{AED}: Antiepileptic drugs
            \item \textbf{BOLD}: Blood oxygenation level dependent
            \item \textbf{CADT}: Computer aided diagnostic tools
            \item \textbf{CL1OCV}: Cyclical leave-one-out cross validation
            \item \textbf{ECD}: Equivalent current dipole
            \item \textbf{EEG}: Electroencephalography
            \item \textbf{FDG-PET}: 18F-Fludeoxyglucose Positron Emission 
Tomography
            \item \textbf{ML}: Machine Learning
            \item \textbf{NNP}: Neurologically normal patients
            \item \textbf{PWE}: Epileptic seizure patients
            \item \textbf{PWN}: Non-epileptic seizure patients
            \item \textbf{RFTC}: Random field theory correction
            \item \textbf{ROI}: Region of interest
            \item \textbf{R/LTLE}: Right/Left Temporal Lobe Epilepsy
            \item \textbf{TBI}: Traumatic Brain Injury
        \end{itemize}

    \section{Concept Jar}
        Concepts floating around...

        \textbf{Complexity and EEG:} How many distinct patterns are there in an 
EEG?
        Fewer number of distinct patterns \ensuremath{\rightarrow} lower 
complexity.
        (Lempel-Ziv Complexity).

   


    \section{Clusting}
        \subsection{Spectral Clustering Tutorial}
    \section{Colocalization} 
        \subsection{Daunizeau--EEG-fMRI information fusion: biophysics and data 
    analysis}
            \textbf{Questions:}
            \begin{itemize}
                \item How is estimating neuronal activity from hemodynamic
response a [difficult] temporal deconvolution problem?
                \item Do spatial priors of fMRI derived activation only consider
activation and (not) activation as the two possible outcomes? Which
probabilities do the prior probability distribution consider?
            \end{itemize}     
            Moving from unimodal recordings to multimodal recordings. 

            \textbf{Background:}``Cerebral activity has man attributes: bioelectric,
            metabolic, hemodynamic, hormonal, endogenous, exogenous, specialized, 
    integrated,
            pathologic, stable, dynamic, etc.'' 

            \begin{itemize}
                \item EEG signals are driven by postsynaptic cortical currents 
    ``associated with
                large pyramidal neurons''
                \item EEG and fMRI data may be mismatched or ``decoupled''
(possibly an indication of some pathology)
                \item Decoupling may be a result of the ``distance between the
neuronal population... generating the EEG signal and the vascular tree... which
provides the blood supply to these neurons,'' among other factors
                \item \textbf{Problem}: Simultaneous EEG/fMRI faces
challenges--low SNR relative to unimodal collection, artifacts in EEG, etc.
                \begin{itemize}
                    \item MRI Scanning 'imaging artifact' (Allen 2000),
producing time-varying magnetic fields that induce currents (and therefore
voltages)
                    \item Cardiac pulsation 'pulse artifact' causing moments in
the head/electrodes or of the blood, changing the Hall-voltage
                \end{itemize}
                \item De-noising algorithms... \footnote{This is looking a lot
like spike sorting}
                \item EEG to fMRI approaches (asymmetric)
                \begin{itemize}
                    \item ``Localize, using fMRI brain regions whose response is
temporally correlated with a given EEG-defined event''
                    \item ``Spontaneous fluctuations of power in specific
frequency bands are quantified in the EEG traces.''
                    \item \textbf{Goal}: ``Linear decomposition of fMRI data
which covaries with a time-frequency decomposition of the EEG''
                \end{itemize}                    
                \item fMRI to EEG (asymmetric)
                \begin{itemize}
                    \item Use spatial prior probability distributions derived
from fMRI to solve the EEG source reconstruction problem\footnote{Does this
assume that fMRI and EEG are strongly coupled?}      
                \end{itemize}
                \item \textbf{Problem}: EEG and fMRI may differ in obtained
localization, so how are the relative importance or weighting of EEG/fMRI
determined?
            \end{itemize}

            \textbf{Insight}: EEG does not ``specify uniquely the location of 
    underlying
            biolectric activity''

    \section{Epilepsy: Common}
        \subsection{Henry--Presurgical epilepsy localization with interictal 
cerebral
        dysfunction}
        This paper documents a variety of scenarios with respect to ictal, 
interictal,
        and peri-ictal hypometabolism in TLE patients. It also contains a of 
issues surrounding the localization of abnormalities and cortical resection. 
These
        scenarios as used to describe the process of using FDG-PET and EEG + MRI 
to
        localize epileptogenetic areas, determine electrode placement for 
monitoring
        these areas, and establishing a prognosis for the resection of these 
areas. 

        \textbf{Background}: ``Neuropsychological evaluation is critical to 
evaluating
        the potential benefit of epilepsy surgery''

        \textbf{Observations}
        \begin{itemize}
            \item ``Bilateral but asymmetric temporal lobe hypometabolism can be
            detected readily with statistical parametric imaging techniques, but 
usually
            appears only as unilateral temporal hypometabolism on visual 
interpretation.''
            \item AEDs may uniformly reduce metabolism.  \item Higher resolution 
better detect hypometabolism.
            \item Hypometabolism is diffuse
            \item Temporal hypometabolism is usually more severe than 
extratemporal
            hypometabolism
            \item Hypometabolism changes with time following seizures
            \item ``Symmetric, severe, bilateral temporal hypometabolism also is 
associated
            with a higher incidence of postoperative seizures''
            \item Infants suffering from West's syndrome that underwent 
resection ``often
            had cessation of seizures, and normal or near-normal cognitive 
development,'' as
            opposed to developing autism
            \item ``Epileptogenetic zone is typically a ... area of cortical 
dysplasia.'' \item ``Impairments of of verbal fluency are common in epilepsy 
interictally''
        \end{itemize}

    \subsection{Kerr--Balancing Clinical and Pathologic Relevance in the Machine 
Learning
    Diagnosis of Epilepsy}
        \textbf{Distinction}: Patients with non-epileptic seizures as control 
versus
        normal patients as control.

        ``Non-epileptic seizures are primarily psychiatric events.'' ``...based 
on
        ruling out all organic causes for the attacks.''


        Common features of Epilepsy: ``Hypersynchronized activity of neurons, 
inhibited
        activity in surrounding tissue, focal cortical atrophy, MRI signal
        hyperintensivity, increased cell death of inhibitory cells, ensuing 
gliosis''

        \textbf{Insight}: Patients with seizures may recieve antiepileptic drugs 
that manipulate
        neural networks and therefore EEGs that differences between NNPs and 
PWEs are
        enhanced.

        \textbf{Observation}: PWNs and PWEs may share many common underlying 
factors: ``PWNs
        model their seizures after those they have seen or heard about...'' and 
``both
        PWN and PWE are associated with traumatic brain injury''

        \textbf{Insight}: Comparison between PWN and PWE models the comparison 
physicians
        perform--that is, if a patient suffers from seizures, is this seizure 
epileptic
        or non-epileptic? This contrasts with trying to determine whether a 
patient is
        epileptic regardless of whether the patient suffers from seizures.

        \textbf{Methodology}: Multilayer Perceptron used to ``discriminate 
between either PWN or
        NNP versuses RTLE or LTLE''. \textbf{minimum redundancy-maximum relvancy 
mRMR of
        MATLAB}.

        \textbf{Observation}: Each binary comparison e.g. LTLE vs. PWN, LTLE vs. 
NNP, RTLE vs.
        NNP, etc. often corresponds to differing ROIs. Even LTLE vs. PWN and 
LTLE vs.
        NNP have differences--PWN and NNP differ from LTLE differently!

        \textbf{Observation}: Despite the differences in ROIs for different 
control
        groups, the ability to diagnose LTLE/RTLE remained similar. Therefore,
        comparing PWEs to NNPs does \textbf{not} artificially increase 
discriminative
        ability!

        \textbf{Take Home}: Let the clinical question determine the control 
group, and
        the question may be a ``pathologic description of disorders'' or the
        ``development of clinically applicable tools.''  

\end{document}
